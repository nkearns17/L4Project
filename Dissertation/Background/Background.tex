\documentclass{article}
\begin{document}
\title{Java Programming Revision Tool}
\author{Nicole Keaarns}
\maketitle
\tableofcontents

\newpage

\section{Background}

This section discusses several different existing artifacts, most relating to Java programming or another programming language. These tools were selected to provide design ideas for my own Java revision application, focusing on how each tool provides the user with the material to learn.

\subsection{Investigation of Exisiting Artifacts}

There are many methods of learning how to program in a new language and there are a huge number of applications available to teach users how to do so. I have broken down the exisiting artefacts into several categories, as shown in the image below.\\
\\
\textbf{INSERT DIAGRAM!!!!!!!!}


\subsubsection{Static Learning Tools}

\begin{itemize}
\item Java Puzzler
\item Oracle
\end{itemize}

\subsubsection{Interactive Tools}

\textbf{\underline{Guided Walkthroughs:}}\\

\begin{itemize}
\item Code Avengers
\item CodeCademy
\item Learn Java Online
\item TryRuby
\end{itemize}

\textbf{\underline{Free-Form Tools:}}\\

\begin{itemize}
\item GroovyConsole
\item Compile Online
\end{itemize}

\subsubsection{Quizzes}

\begin{itemize}
\item Bitesize
\item ProProfs
\item Indiabix
\end{itemize}

\subsubsection{Community-based Support}

\begin{itemize}
\item Java Programming Community
\item StackOverflow
\item Wikipedia
\end{itemize}

\end{document}
